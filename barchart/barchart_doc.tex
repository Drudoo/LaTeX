\documentclass[]{article}
\usepackage{barchart}

\title{\texttt{barchart}: Easy Bar Chart in \LaTeX
	\medskip\\
	\large Version 0.1.0
}
\author{Frederik Vanggaard}
\date{\today}

\begin{document}

\maketitle

\section{Introduction}
\texttt{barchart} is a \LaTeX\space package for creating simple and easy to use bar-chart. The package is heavily inspired by \texttt{bchart} by Tobias Kuhn. \texttt{barchart} provides an easy way to create vertical barcharts using Tikz.  

\section{Charts}
\texttt{barchart} comes with a variety of customization options for both the overall chart and for each individual bar. A simple chart can be done like this:

\begin{quote}\small
\begin{verbatim}
\begin{barchart}
    \barc{5}
    \barc{7}
    \barc{3}
\end{barchart}
\end{verbatim}
\end{quote}
\begin{quote}\small
\begin{figure}[ht]
	\begin{barchart}
		\barc{5}
		\barc{7}
		\barc{3}
	\end{barchart}
\end{figure}
\end{quote}

The only arguments that are mandatory is the bar value. The charts can be changed using a few simple options such as \texttt{min}, \texttt{max} and \texttt{step}.

\begin{quote}\small
\begin{verbatim}
\begin{barchart}[min = 0, max = 60, step = 10, height=0.25em]
    \barc{30}
    \barc{20}
    \barc{50}
\end{barchart}
\end{verbatim}
\end{quote}
\begin{quote}\small
\begin{figure}[ht]
	\begin{barchart}[min = 0, max = 60, step = 10, height=0.25em]
		\barc{30}
		\barc{20}
		\barc{50}
	\end{barchart}
\end{figure}
\end{quote}

\texttt{height} is used to set the height of the whole chart. Omitting a height on charts with large \texttt{max} values will make the chart enormous.

\section{Bars}
Each bar can be customized to your liking. This can be done using \texttt{color}, \texttt{width} and \texttt{plain}.
\texttt{color} changes the color of each bar like:


\begin{quote}\small
\begin{verbatim}
\begin{barchart}[step = 2]    
    \barc[color=yellow!60]{5}
    \barc[color=orange!60]{7}
    \barc[color=cyan!60]{3}
\end{barchart}
\end{verbatim}
\end{quote}
\begin{quote}\small
\begin{figure}[ht]
	\begin{barchart}[step = 2]
		\barc[color=yellow!60]{5}
		\barc[color=orange!60]{7}
		\barc[color=cyan!60]{3}
	\end{barchart}
\end{figure}
\end{quote}

\texttt{width} changes the width of each bar:

\begin{quote}\small
\begin{verbatim}
\begin{barchart}[step = 2]
    \barc[width=40pt]{5}
    \barc[width=30pt]{7}
    \barc[width=50pt]{3}
    \barc[width=20pt]{3}
\end{barchart}
\end{verbatim}
\end{quote}
\begin{quote}\small
\begin{figure}[ht]
	\begin{barchart}[step = 2]
		\barc[width=40pt]{5}
		\barc[width=30pt]{7}
		\barc[width=50pt]{3}
		\barc[width=20pt]{3}
	\end{barchart}
\end{figure}
\end{quote}
Using less than \texttt{20pt} is not recommended. Instead of using pre-defined step as the examples have shown so far, \texttt{steps} can be used.

\begin{quote}\small
\begin{verbatim}
\begin{barchart}[steps = {1,3,5,7,10}]
    \barc{5}
    \barc{7}
    \barc{3}
\end{barchart}
\end{verbatim}
\end{quote}
\begin{quote}\small
\begin{figure}[ht]
	\begin{barchart}[steps = {1,3,5,7,10}]
		\barc{5}
		\barc{7}
		\barc{3}
	\end{barchart}
\end{figure}
\end{quote}

\texttt{plain} is a boolean which can either be \texttt{true} or \texttt{false} and removes the label on the bar.
\begin{quote}\small
\begin{verbatim}
\begin{barchart}
    \barc{5}
    \barc[plain=true]{7}
    \barc[plain=true]{3}
\end{barchart}
\end{verbatim}
\end{quote}
\begin{quote}\small
\begin{figure}[ht]
	\begin{barchart}
		\barc{5}
		\barc[plain=true]{7}
		\barc[plain=true]{3}
	\end{barchart}
\end{figure}
\end{quote}


\section{Skips}

\section{Labels}

\section{Scaling}

\section{Known Issues}

\end{document}